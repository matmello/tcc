\chapter[Metodologia de Pesquisa]{Metodologia de Pesquisa}

Uma pesquisa pode ser classificada de acordo com os procedimentos técnicos utilizados:

Pesquisa bibliográfica é desenvolvida a partir de material já elaborado,
constituído principalmente de livros e artigos científicos. Embora em quase todos
os estudos seja exigido algum tipo de trabalho desta natureza, há pesquisas desenvolvidas
exclusivamente a partir de fontes bibliográficas. Parte dos estudos exploratórios
podem ser definidos como pesquisas bibliográficas, assim como certo
número de pesquisas desenvolvidas a partir da técnica de análise de conteúdo \cite{gil}

A pesquisa documental assemelha-se muito à pesquisa bibliográfica. A única
diferença entre ambas está na natureza das fontes. Enquanto a pesquisa bibliográfica
se utiliza fundamentalmente das contribuições dos diversos autores sobre
determinado assunto, a pesquisa documental vale-se de materiais que não receberam
ainda um tratamento analítico, ou que ainda podem ser reelaborados de
acordo com os objetivos da pesquisa.  \cite{gil}

Pesquisa experimental consiste em determinar um
objeto de estudo, seleciona-se as variáveis que seriam
capazes de influenciá-lo, define-se as formas de
controle e de observação dos efeitos que a variável
produz no objeto. \cite{gil}

Levantamento é a interrogação direta das pessoas
cujo comportamento se deseja conhecer. Procede-se à
solicitação de informações a um grupo significativo de
pessoas acerca do problema estudado para, em seguida,
mediante análise quantitativa, obterem-se as conclusões
correspondentes aos dados coletados. \cite{gil}

Estudos de
campo procuram o aprofundamento das questões propostas. O planejamento do
estudo de campo apresenta uma grande flexibilidade, podendo ocorrer mesmo que seus objetivos sejam reformulados ao
longo do processo de pesquisa \cite{gil}

Estudo de caso consiste no estudo profundo e
exaustivo de um ou poucos objetos, de maneira
que permita seu amplo e detalhado
conhecimento \cite{gil}

O Estudo de Caso é um estudo empírico que
investiga um fenômeno atual dentro do seu contexto de realidade, quando as
fronteiras entre o fenômeno e o contexto não são claramente definidas e no qual
são utilizadas várias fontes de evidência \cite{yin}

A pesquisa-ação é um tipo de pesquisa social que é concebida e
realizada em estreita associação com uma ação ou com a resolução
de um problema coletivo e no qual os pesquisadores e os participantes
representativos da situação da realidade a ser investigada estão
envolvidos de modo cooperativo e participativo. \cite{thiollent}

Agora quanto os resultados, existem pesquisas exploratórias que são desenvolvidas com o objetivo de proporcionar visão
geral, de tipo aproximativo, acerca de determinado fato. Este tipo de pesquisa
é realizado especialmente quando o tema escolhido é pouco explorado e torna-se
difícil sobre ele formular hipóteses precisas e operacionalizávei.
A pesquisa descritiva têm como objetivo primordial a descrição das características
de determinada população ou fenômeno ou o estabelecimento de
relações entre variáveis. São inúmeros os estudos que podem ser classificados sob
este título e uma de suas características mais significativas está na utilização de
técnicas padronizadas de coleta de dado. E enfim a pesquisa explicativa que têm como preocupação central identificar os fatores
que determinam ou que contribuem para a ocorrência dos fenômenos. Este
é o tipo de pesquisa que mais aprofunda o conhecimento da realidade, porque explica a razão, o porquê das coisas.
Por isso mesmo é o tipo mais complexo e delicado, já que o risco de cometer erros aumenta consideravelmente.  \cite{gil}

Considerando os procedimentos técnicos utilizados neste trabalho, foi utilizado um Estudo de Caso, apresentando apenas um
contexto específico de atuação do software desenvolvido, o par ouro-dólar (XAU/USD). O que apenas fornece indícios do funcionamento
do software estando altamente acoplado a este contexto pré-determinado.

\section[Atividades]{Atividades}

As atividades deste trabalho foram divididas conforme a tabela seguinte


\begin{table}[h]
	\centering
	\caption{Atividades Planejadas}
	\label{tab01}
  \begin{center}
      \begin{tabular}{ | l | p{5cm} | p{5cm}}
      \hline
      Número & Atividade & Descrição \\ \hline
      1. & Descrever Objetivos e Organização do trabalho & Nesta atividade será definida a estrutura do trabalho assim como os objetivos aos quais este trabalho visa atingir\\ \hline
      2. & Buscar Material para Referência & Atividade compreende toda a busca por material relevante a temática do presente trabalho \\ \hline
      3. & Construir Referencial Teórico & A partir do material recolhido na atividade 2, o referencial teórico é elaborado, contendoo informações cruciais para o pleno entendimento deste projeto \\ \hline
      \end{tabular}
  \end{center}
\end{table}

\begin{table}[h]
	\centering
	\caption{Continuação das Atividades Planejadas}
	\label{tab02}
  \begin{center}
      \begin{tabular}{ | l | p{5cm} | p{5cm}}
      \hline
      Número & Atividade & Descrição \\ \hline
			4. & Selecionar tipo de rede neural & Nesta atividade o tipo de rede neural a ser utilizado pelo software é escolhida \\ \hline
			5. & Selecionar entradas do sistema & É escolhido um conjunto de entradas para servirem de insumo ao software desenvolvido \\ \hline
			6. & Desenhar Software & É realizado o desenho do software proposto, desde sua arquitetura a implementação dos nós \\ \hline
			7. & Desenvolver Software & Esta atividade engloba todo o processo de desenvolvimento do software BrainBot \\ \hline
			8. & Medir Efetividade & É mensurada a efetividade do software desenvolvido em prever mudanças de tendência no contexto definido \\ \hline
			9. & Concluir & Os resultados obtidos são reportados e a conclusão é evidenciada  \\ \hline
      10. & Redesenhar & O software é redesenhado, utilizando outras variáveis complementares e aperfeiçoando os nós desenvolvidos  \\ \hline
      11. & Retrabalho & O desenho gerado na atividade 10 é então contruído  \\ \hline
      12. & Medir e Concluir & Novas medições são feitas e a conclusão é complementada \\ \hline
      \end{tabular}
  \end{center}
\end{table}



\section[Cronograma]{Cronograma}

As atividades foram divididas em Sprints seguindo o cronograma seguinte:

\begin{table}[h]
	\centering
	\caption{Cronograma}
	\label{tab03}
  \begin{center}
      \begin{tabular}{ | l | l | p{5cm} | p{5cm}}
      \hline
      Sprint & Atividades & Início & Fim \\ \hline
      1 & 1, 2 e 3 & 29/03/2017 & 29/04/2017 \\ \hline
      2 & 4 e 5 & 30/04/2017 & 15/05/2017 \\ \hline
      3 & 6 & 16/05/2017 & 31/05/2017 \\ \hline
      4 & 7 & 01/06/2017 & 30/06/2017 \\ \hline
      5 & 8 e 9 & 01/07/2017 & 17/07/2017 \\ \hline
      6 & 10 & 08/08/2017 & 31/08/2017 \\ \hline
      7 & 11 & 01/09/2017 & 13/08/2017 \\ \hline
      8 & 12 & 14/08/2017 & 01/11/2017 \\ \hline
      \end{tabular}
  \end{center}
\end{table}





% \subsection{Objetivos Gerais}
% \subsection{Objetivos Específicos}
% \section{Metodologia}
% \section{Organização}

 % \lq\lq \citeonline{bordalo1989}
