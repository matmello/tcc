\begin{resumo}
  Este trabalho teve como objetivo principal o desenho de uma rede neural
  recorrente, sendo implementada no desenvolvimento do software BrainBot, que
  busca prever transições de tendência no Mercado de Moedas. Foi realizado um
  estudo de caso evidenciando os indícios do funcionamento da rede neural
  proposta num contexto específico do Mercado de Moedas, o par de negociação
  ouro-dólar (XAU/USD). Para o desenvolvimento do software, foram utilizadas técnicas
  do SCRUM, com a separação das funcionalidades em features, histórias de
  usuário e tarefas. Foram abordados os tipos de redes neurais e a escolha
  da rede neural recorrente, assim como as entradas a serem utilizadas
  no software. Os resultados da aplicação da rede neural foram então registrados
  em forma de percentual de acerto, tendo como base verificações a posteriori
  \textit{(backtesting)}. Foram implementados testes unitários e de integração,
  considerando a rigorosidade do contexto de funcionamento do software e enfim
  foi relatada a conclusão.

 \vspace{\onelineskip}

 \noindent
 \textbf{Palavras-chaves}: Mercado de Moedas, SCRUM, Testes de Software, Redes
 Neurais.
\end{resumo}
