\chapter[Introdução]{Introdução}

O mercado de moedas baseia-se em um conceito muito simples, a troca de uma moeda
por outra, muitas vezes chamada de conversão. Mercado de moedas também se refere
ao mercado global onde moedas são trocadas virtualmente.

No ano de 1875 foi acordado e entrou em vigor o padrão-ouro, que significava que o
valor da moeda de cada país seria baseado no valor da reserva de ouro de cada respectivo
país, buscando que o governo pudesse garantir a conversão da moeda em alguma porção
de ouro. Esse padrão foi extremamente importante, uma vez que foi o primeiro padrão
a ser adotado mundialmente como sistema monetário, deixando o valor de cada moeda mais
bem definido e simples. Na teoria o padrão era bom, mas depois das guerras mundiais,
os países que já estavam estressados economicamente começaram a imprimir cédulas sem controle,
fazendo com que o padrão-ouro perdesse o sentido, uma vez que a quantidade de moeda existente
não era mais compatível com a reserva de ouro associada a um determinado país.

Em 1944 entrou em vigência outro acordo monetário, que funcionava da mesma maneira do padrão-ouro,
mas desta vez em vez de usar ouro como padrão de controle, o valor de cada moeda seria agora baseado
no dólar, e o dólar seria a única moeda que permaneceria sendo baseada em ouro. Foi após esse acordo
que o Banco Mundial foi criado, uma instituição gerenciada pela Organização das Nações Unidas. As reservas
de ouro dos Estados Unidos começaram a cair drasticamente e em 1971 este declarou que não seria mais possível
trocar o dólar americano por ouro, o que fez cair por terra mais uma vez, o acordo monetário existente na época.

Desde então, o mundo está utilizando taxas de câmbio flutuante, atualmente conhecido como Mercado de Moedas, onde
o valor de uma dada moeda é baseado na oferta e na procura comparado com outras moedas. Bem diferente de taxas de câmbio fixas,
onde o governo tem a autoridade de determinar o valor da moeda \cite{investopedia}.

Esse novo regime monetário que está vigente até os dias atuais dá margem a oscilações consideráveis durante as negociações, uma vez
que cada moeda não possui um valor fixo estipulado, é nesse ponto que o presente trabalho se baseia e busca utilizar de técnicas de
inteligência artificial para realizar previsões no Mercado de Moedas. Foi considerado o desenvolvimento de um agente expert capaz de prever
alterações de tendência que busque responder a seguinte questão: É possível desenhar e contruir um software que utilize técnicas de
inteligência artificial para prever mudanças de movimentos no mercado de moedas?



\section{Objetivos}

O objetivo deste relatório é descrever as atividades que foram realizadas
durante a realização do estágio obrigatório na Happe Soluções em Tecnologia,
localizada em Brasília, DF.

% \section{Motivações}

% \subsection{Objetivos Gerais}
% \subsection{Objetivos Específicos}
% \section{Metodologia}
% \section{Organização}

 % \lq\lq \citeonline{bordalo1989}
